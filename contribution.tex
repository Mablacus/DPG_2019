\documentclass{scdpg}
\begin{document}
\scBookLanguage{de}
\begin{scAbstract}
\scNoUseTeX
\scLanguage{en}
\scTitle{Multimode strong coupling of laser-cooled atoms to a nanofiber-based ring resonator}
\scAuthor{*}{Martin}{Blaha}{1}
\scAuthor{}{Aisling}{Johnson}{1}
\scAuthor{}{Alexander}{Ulanov}{2}
\scAuthor{}{Jürgen}{Volz}{1}
\scAuthor{}{Philipp}{Schneeweiss}{1}
\scAuthor{}{Arno}{Rauschenbeutel}{3}
\scAffiliation{1}{Atominstitut, TU Wien, 1020 Wien, Austria}
\scAffiliation{2}{Russian Quantum Center, 143025 Moscow, Russia}
\scAffiliation{3}{Institut für Physik, Humboldt-Universität zu Berlin, 10099 Berlin, Germany}
\scBeginText
We report on the observation of multimode strong coupling between a cloud of cold atoms and a nanofiber-based fiber ring resonator. This novel regime of CQED can be reached when the collective coupling strength $g$ between the atoms and the light field exceeds the free spectral range FSR of the resonator, leading to coupling of the emitters with more than one longitudinal resonator mode[1].
In our cavity an exceptionally small free spectral range of 7.1~MHz, can be reached by using a 30~m long fiber ring resonator[2], with an integrated optical nanofiber of subwavelength-diameter waist.
%, that evanescently couples to a cloud of laser cooled Cesium atoms.
The experimental signature of this regime lies in the transmission spectrum of the loaded cavity, which we measured for increasing couplings until values as large as g = 2 x FSR.
Furthermore, we characterise the experimental platform by measuring second-order correlations at the output of the resonator.
This photon-statistics contain information on the number of atoms coupled to the cavity as well as evidence of the light-atom interplay in the resonator.
[1] Meiser, D., and P. Meystre, Physical Review A 74.6 (2006): 065801.
[2] Schneeweiss, Philipp, et al., Optics letters 42.1 (2017): 85-88.

%what shall i reference?
%[] V. Bužek, G. Drobny, M. G. Kim, M. Havukainen \& P. L. Knight, Numerical simulations of atomic decay in cavities and material media, Phys. Rev. A 60, 582 (1999).
%[] D.O. Krimer, M. Liertzer, S. Rotter \& H.E. Türeci, Route from spontaneous decay to complex multimode dynamics in cavity QED, Phys. Rev. A 89, 33820 (2014).

\scEndText
\scConference{Rostock 2019}
\scPart{Q}
\scContributionType{Vortrag;Talk}
\scTopic{7.2 Quanteneffekte (Resonator QED); 7.2 Quantum Effects (Cavity QED)}
\scEmail{martin.blaha@tuwien.ac.at}
\scCountry{Austria}
\end{scAbstract}
\end{document}
