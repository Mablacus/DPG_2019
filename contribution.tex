\documentclass{scdpg}
\begin{document}
\scBookLanguage{de}
\begin{scAbstract}
%\scNoUseTeX
\scLanguage{en}
\scTitle{Multimode strog coupling of laser-cooled atoms to a nanofiber-based ring resonator}
\scAuthor{}{Martin }{Blaha}{1}
\scAuthor{}{Aisling}{Johnsons}{1}
\scAuthor{}{Alexander }{Ulanov}{2}
\scAuthor{}{Jürgen}{Volz}{1}
\scAuthor{}{Philipp}{Schneeweiss}{1}
\scAuthor{}{Arno}{Rauschenbeutel}{3}
\scAffiliation{1}{Atominstitut, TU Wien, Stadionallee 2, 1020 Wien}
\scAffiliation{2}{Russian Quantum Center, 100 Novaya Street, Skolovo, Moscow 143025, Russia}
\scAffiliation{3}{Department of Physics, Humboldt-Universität zu Berlin, 10099 Berlin, Germany}
\scBeginText
Cavity quantum electrodynamics (CQED) have developed from the textbook strong coupling between a single atom and photon to a large community with a variety of experiments and systems. 
We report on the observation of multimode strong coupling between a cloud of cold atoms and and a nanofiber-based fiber ring resonator. In this cavity with an exceptionally small free spectral range (7 MHz), the collective coupling between the atoms and the light field exceed this frequency scale, leading to coupling of the emitters with more than one longitudinal resonator mode. The experimental signature of this regime lies in the transmission spectrum of the loaded cavity, which we measured for increasing couplings until values as large $g = 2 \Delta_{FSR}$. We also further characterise our novel experimental platform by measuring second-order correlations at the output of the resonator: the statistics of the photons at the output contain information on the number of atoms coupled in the cavity as well as evidence of the light-atom interplay in the resonator.

An ensemble of laser-cooled Cesium (Cs) atoms is interfaced with an optical nanofiber,
itself contained in a 30 m long fiber ring resonator [1] with a free spectral range (FSR) of 7
MHz. The collective coupling of the atomic cloud exceeds the FSR, leading to multimode
strong coupling (or superstrong coupling) [2-4], a novel regime of light-matter interaction.

Interfacing atoms with light propagating through an optical nanofiber is a promising method for the study of light-matter interaction. We are currently developing an experiment based on such a nanofiber to probe new regimes of cavity quantum electrodynamics (cQED).
To do so, we experimentally realize an optical fiber ring resonator that includes a tapered section with a subwavelength-diameter waist. In this section, the guided light exhibits a significant evanescent field which allows for efficient interfacing with optical emitters.


The tapered nanofiber, including the 400 nm thick waist, is integrated into a 30m long fiber ring resonator, where a commercial tunable fiber beam splitter provides simple and robust coupling to the resonator. Key parameters of the resonator such as the out-coupling rate, free spectral range, and birefringence can be adjusted. Thanks to the low taper- and coupling-losses, the resonator exhibits an unloaded finesse of 𝐹=28±1 mainly given by losses in splices and propagation through the fiber.
% This is about the length of teh abstract 
\\
The nanofiber section of the resonator is brought into vacuum and by creating a MOT cloud around the waist, we can couple an ensemble of Cesium atoms to the cavity field which propagates as an evanescent field in this region, where the cross-section of the fiber is smaller than the wavelength of light. The presence of many atoms that simultaneously couple to the resonator field allows for a collective coupling strength g that exceeds the free spectral range of a few MHz of the fiber ring resonator. Based on preliminary measured parameters, we estimate that the system reaches before mentioned regime and therefore serves as a platform for investigating optical multimode strong coupling experiments. 
Finally, we discuss the possibilities of using the resonator for applications based on chiral quantum optics. Notably, the polarization properties of light in the tapered fiber can lead to chiral, i.e propagation-direction dependent, light-matter interaction, which is expected to strongly affect the collective behaviour of the atomic cloud in the cavity field. In general, this setup should lend itself well to novel cQED experiments as well as quantum simulation of strongly coupled light and matter.

Optical nanofibers have already successfully been used to investigate light-matter
interaction at the single-photon level: emitters such as neutral atoms can be addressed
with the guided field since a significant fraction of the propagating light is contained in an
evanescent field surrounding the fiber [5, 6]. They offer a versatile platform where novel
regimes are accessible. Their excellent transmissions exceeding 99% [5, 6] permits their
integration into cavities to reach the strong coupling regime of CQED [7, 8]. In this work we
include a nanofiber into a 30 meter-long fiber ring resonator. The inherent fiber integration
makes the system very flexible and allows us to tune key parameters such as the free
spectral range (FSR) and eigenpolarizations. Due to the strong enhancement of lightmatter
interaction in the optical nanofiber, several hundred atoms suffice to achieve a
collective coupling strength that exceeds the FSR of 7 MHz of the cavity.

[1] P. Schneeweiss, S. Zeiger, T. Hoinkes, A. Rauschenbeutel \& J. Volz, Fiber ring
resonator with nanofiber section for chiral cavity quantum electrodynamics and
multimode strong coupling, Opt. Lett. 42, 85 (2017).
[2] D. Meiser \& P. Meystre, Superstrong coupling regime of cavity quantum
electrodynamics, Phys. Rev. A 74, 065801 (2006).
[3] V. Bužek, G. Drobny, M. G. Kim, M. Havukainen \& P. L. Knight, Numerical simulations
of atomic decay in cavities and material media, Phys. Rev. A 60, 582 (1999).
[4] D.O. Krimer, M. Liertzer, S. Rotter \& H.E. Türeci, Route from spontaneous decay to
complex multimode dynamics in cavity QED, Phys. Rev. A 89, 33820 (2014).
[5] E. Solano, J. A Grover, J.E Hoffmann, S. Ravets, F. K. Fatemi, L. A. Orozco \& S. L.
Rolston, Optical Nanofibers: a new platform for quantum optics, arXiv: 1703.10533v1.
[6] T. Nieddu, V. Gokhroo, S. Nic Chormaic, Optical nanofibers and neutral atoms,
arXiv:1512.02753.
[7] S. Kato \& T. Aoki, Strong coupling between a trapped single atom and an all-fiber
cavity, Phys. Rev. Lett. 115, 093603 (2015)
[8] S. K. Ruddell, K. E. Webb, I. Herrera, A. S. Parkins \& M. D. Hoogerland, Collective
Strong Coupling of cold aotms to an all-fiber ring cavity, Optica 4, 000576 (2017)
[9] T. W Hänsch \& B. Couillaud, Laser frequency stabilization by polarization spectroscopy
of a reflecting reference cavity, Optics Communications 35, 3 (1980).
[10] V. Torggler, S. Krämer \& H. Ritsch, Quantum annealing with ultracold atoms in a
multimode optical resonator, Phys. Rev. A 95, 32310 (2016).


\scEndText
\scConference{Rostock 2019}
\scPart{Q}
\scContributionType{Vortrag;Talk}
\scTopic{7.2 Quanteneffekte (Resonator QED); 7.2 Quantum Effects (Cavity QED)}
\scEmail{martin.blaha@tuwien.ac.at}
\scCountry{Austria}
\end{scAbstract}
\end{document}
