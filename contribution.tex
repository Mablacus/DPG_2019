\documentclass{scdpg}
\begin{document}
\scBookLanguage{de}
\begin{scAbstract}
%\scNoUseTeX
\scLanguage{en}
\scTitle{Observation of Multimode strong coupling of laser-cooled atoms to fiber-guided photons}
\scAuthor{*}{Martin}{Blaha}{1}
\scAuthor{}{Aisling}{Johnson}{1}
\scAuthor{}{Alexander}{Ulanov}{2}
\scAuthor{}{Jürgen}{Volz}{1}
\scAuthor{}{Philipp}{Schneeweiss}{1}
\scAuthor{}{Arno}{Rauschenbeutel}{3}
\scAffiliation{1}{Atominstitut, TU Wien, 1020 Wien, Austria}
\scAffiliation{2}{Russian Quantum Center, 143025 Moscow, Russia}
\scAffiliation{3}{Institut für Physik, Humboldt-Universität zu Berlin, 10099 Berlin, Germany}
\scBeginText
We report on the observation of multimode strong coupling between a cloud of cold atoms and a nanofiber-based fiber ring resonator.
This novel regime of CQED is reached when the collective coupling strength g between a cloud of laser-cooled Cesium atoms and the light field exceeds the free spectral range (FSR) of the resonator, leading to coherent coupling of the atoms with more than one longitudinal resonator mode simultaneously [1].
We can do that by using a 30 m long fiber ring resonator with an integrated optical nanofiber where the mode cross section is independent of the resonator length. This leads to an exceptionally small free spectral range of 7.1 MHz [2].
The measured transmission spectrum provides clear experimental evidence for multimode strong coupling of the loaded cavity, measured for increasing couplings until values as large as g = 2FSR.
In this regime of CQED atoms can mitigate interactions between photons in different resonator modes.
We envision to make use of this to e.g., implement novel non-classical photonic states.
%Furthermore, we characterise the experimental platform by measuring second-order correlations at the output of the resonator.
%The photon-statistics contain information on the number of atoms coupled to the cavity as well as evidence of the light-atom interplay in the resonator.

[1] D. Meiser et al., Phy. Rev. A 74, (2006).

[2]  P. Schneeweiss et al., Opt. Lett. 42, (2017).
\scEndText
\scConference{Rostock 2019}
\scPart{Q}
\scContributionType{Vortrag;Talk}
\scTopic{7.2 Quanteneffekte (Resonator QED); 7.2 Quantum Effects (Cavity QED)}
\scEmail{martin.blaha@tuwien.ac.at}
\scCountry{Austria}
\end{scAbstract}
\end{document}
